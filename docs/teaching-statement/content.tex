\section*{Teaching Preferences}

Teaching Computer Science is in general exciting to me. My background
makes me an ideal fit to teach graduate and undergraduate courses in
Programming Languages, Formal Methods, Software Engineering,
Databases, Distributed Systems, and Operating Systems. I would like to
contribute to their curriculum drawing from my research experience at
the intersection of these diverse fields. Software Engineering, for
example, could be enriched with lectures and exercises on applying
state-of-the-art program analyses on open-source databases,
distributed systems, and web applications readily available on Github.
The access to full provenance -- commit history, bug reports, their
resolutions, and developer comments, makes it possible to compare the
results of the analyses on multiple versions of the source code, and
determine their effectiveness in detecting the bugs known to be real
or false.  This would let students understand the shortcomings of the
state-of-the-art in addressing the real-world problems, and prepare
them for further research in Software Engineering. Likewise, I would
extend the curriculum of Systems courses with recent developments in
trustworthy sofware systems, such as Unikernels, IronFleet, seL4, and
FSCQ File System. A graduate-level Programming Languages course, I
believe, would benefit from an emphasis on automated reasoning
wherever possible. While I do see the value in teaching students how
to write certified programs and meta-theoretic proofs (normalization,
type safety etc), I don't think making them labor through most mundane
proofs in Coq is the way to go. Instead, I would restructure any such
course around proof systems that put automation first, such as F*,
Dafny, Lean, and Liquid Haskell. 
% I would encourage students to think about automation \emph{in
% conjunction} with verfication and theorem proving, for my experience
% suggests that a consideration of former would significantly alter
% one's approach towards later.

I also have ideas for new courses that prepare graduate students for
inter-disciplinary research with other branches of science and
humanities. In particular, I am most excited about a seminar course
that explores the possibility of using programs as a means of
capturing human knowledge and thought process in the fields of
Medicine, Law, and Economics. I am inspired by the recent successes of
the pioneering work done by our fellow computer scientists in this
direction. Examples include automated drug discovery by Might et al's
MediKanren (OOPSLA'19 Keynote-1), Tax loopholes detection through
Formal Methods by Lawsky et al (POPL'18 Keynote-3), and inviolable
economic ``smart'' contracts on the emerging Blockchain
infrastructure. I think it is worthwhile to institute a course to
focus on such diverse applications of Computer Science so as to
broaden the perspective of our graduate students as they decide on
their thesis topics.

\section*{Teaching Methodology}

In this section, I shall explain two core principles behind my
teaching methodology.

{\itshape\color{MidnightBlue} Utility over aesthetics; problems over
solutions} In my observation, the common motivation for students to
enroll in a course is to acquire the skills that they believe will
help them in their career in research or industry. It is therefore
understandable that if a student fails to see a clear causal
relationship between a concept taught in the class and her
professional success, she would disengage from the learning process. I
believe there has to be a compelling utility of what is taught in the
class in the context of students' professional journey. Such problem-
and utility-guided approach to teaching is, I believe, more productive
than the alternative approaches that stress on solutions, concepts and
their aesthetics.

{\itshape\color{MidnightBlue} Rigor over breadth} A CS instructor
should strike a right balance between the breadth of a course
curriculum, and the rigor with which the topics are taught in the
class. When in quandary, I believe it is better to err on the side of
rigor at the expense of breadth for rigor begets clarity, and clarity
is a necessary precondition to transform passive subject knowledge
into an active skill when an opportunity presents itself. Thus it is
imperative to prioritize rigor over breadth in our teaching if we were
to produce students who are highly skilled and not merely
knowledgeable.

\section*{Teaching Experience}

As a Visiting Assistant Professor at Purdue, I teach CS307 Software
Engineering to undergraduates this (Fall'19) semester, which gives me
an opportunity to test some of my ideas and beliefs in practice. The
curriculum I designed (with significant inputs from my co-instructor
and the previous instructors) includes Agile practices, basic and
advanced testing techniques, Symbolic and Concolic execution
strategies, a primer on concurrency bugs, and Gradual Typing. The
major evaluation component of the course is a semester-long software
engineering project, where students develop a large-scale web
application in three ``sprints''. I have therefore structured my
instruction to relate every concept taught in the class to real
problems developers face in building large-scale web applications
today. For instance, the lecture on concurrency bugs includes code
examples from web applications, and a demonstration of how they can be
exploited by hackers to inflict a financial loss (inspired by
Warszawski et al, SIGMOD'17). Likewise the lecture on Gradual Typing
includes examples contrasting Javascript from its gradually-typed
variant -- Typescript, and a demonstration of how the latter benefits
frontend web developers. While the formal feedback is still pending,
the informal feedback I got from students so far has been extremely
encouraging.

\section*{Mentorship Experience}

During my graduate studies at Purdue, I have had the opportunity to
mentor junior graduate students who later became my co-authors. On
four of my research papers, my immediate co-author (2nd or 1st) is a
junior graduate student for whom it was the first ever publication.
Beyond one-to-one mentorship, I have also taken initiatives for
collective professional advancement that have stood the test of time.
PurPL, a Purdue PL reading group I started in Fall 2013, is still
active today\footnote{PurPL matured into a larger research
organization at Purdue, but that is largely due to the efforts of
others.}. I will be sure to take more such initiatives in the future
as I enjoy sharing the excitement of learning and discovery with my
fellow researchers.
