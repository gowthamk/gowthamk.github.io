In a staunch defense of his life's work in pure mathematics, Prof. G H
Hardy writes in \emph{A Mathematician's Apology} that the ``real''
mathematics is almost wholly ``useless'' whereas useful mathematics is
``intolerably dull''.  A real mathematician therefore ought to indulge
in abstract mathematics, which in Hardy's opinion has the most
aesthetic value albeit being devoid of any worldly purpose. A
physicist's real world, Hardy contests, is merely a model of the
mathematician's reality, which is more general and all-encompassing.
The pure mathematician therefore need not be constrained by physical
facts, especially because, as Hardy argues, ``imaginary universes are
so much more beautiful than this stupidly constructed real one''.
Hardy's unapologetic disdain for any materialistic application of
mathematics is patently evident in his book \emph{A Course of Pure
Mathematics}, where the reader is hard-pressed to find a single
demonstration involving a real-world circumstance.

\section*{Teaching Philosophy}

Research in Computer Science is not very much unlike the research in
mathematics in that it too involves building and reasoning about
elegant abstractions, such as the lambda calculi and the petri nets.
It is therefore not inconceivable for us to sympathize with Hardy, and
de-emphasize the real-world utility of our abstractions in favor of
the study of the abstractions themselves. However, I believe that our
sympathies with Hardy must lie strictly within the confines of our own
research and not permeate our teaching, for I believe it is extremely
counterproductive to teach an abstract concept untethered from its
real-world motivation. This leads me to the first tenet of my teaching
philosophy: \emph{utility over aesthetics; problems over solutions}.
In my observation, the common motivation for students to enroll in a
course is to acquire the skills that they believe will help them
advance their career. It is therefore understandable that if a student
fails to see a clear causal relationship between a concept taught
in the class and her professional success, she would disengage from
the learning process, notwithstanding the constant exhortations about
the aesthetic value of the concept. It is therefore imperative that a
student sees a compelling utility of what is taught in the class in
the context of her professional journey -- a problem to which she
\emph{has to} know the solution, before she is presented with a
solution and the overarching concept. Such problem- and utility-guided
approach to teaching is, I believe, more productive than the
alternative approaches that stress on solutions, concepts and their
aesthetics.

The second major tenet of my teaching philosophy is to prioritize
\emph{clarity over truth}, i.e., to sacrifice details, or even
\emph{lie to students}~\cite{lietokids}, if it helps them develop
intuition and clear mental models for complex (computational)
phenomena. The mental models thus developed may be incomplete, or even
(partially) incorrect, but insofar as they help students internalize
and reason about a \emph{subset} of the complex phenomena correctly, I
would consider them useful. It is preferable to gradually refine
coarse intuitions and build on simple-but-incomplete mental models,
rather than to commit to a hard-to-digest truth right from the outset
and confuse students. This opinion is admittedly controversial, yet I
am not alone in harboring it for I find support from influential
thinkers and teachers who share similar
sentiments~\cite{ladder,lietokids, ttt}.

Note that de-emphasizing truth in favor of clarity does not give one
the license to hand-wave the details and be lax in the presentation;
quite the opposite in fact. The third tenet of my teaching philosophy
is the belief that \emph{rigor begets clarity}. A case in the point is
a mathematical proof. When one finds a proof hard to understand, it is
likely the case that it skips supposedly trivial details that turn out
to be non-trivial, and makes intellectual leaps that are too large for
one's mind to bridge. In other words, the proof lacks rigor. Similar
case can be made of concepts discussed regularly in CS classrooms. A
CS instructor may prioritize comprehension over correctness and choose
to not discuss an algorithm in its full generality, but I believe that
the purpose of comprehension is not met unless the simplified
algorithm is discussed with a sense of rigor appropriate for the
class. In other words, rigor is a necessary precondition for clarity.

The fourth and the final tenet of my teaching philosophy is the belief
that \emph{inclusion should not be an afterthought}. Thanks to the
K-12 outreach of CS departments around the world, the enrollment of
underrepresented groups in CS is steadily increasing.  While the
outreach efforts can \emph{bring} students from disadvantaged sections
to the classroom, I believe it is the responsibility of a CS
instructor to \emph{keep} them in the classroom by making them feel
included. Fortunately, commitment to inclusion does not necessarily
entail an overhaul of one's teaching methodology; simple gestures
often make a significant difference. For instance, choosing examples
carefully to avoid propagating the stereotype threat in classrooms has
been known to be quite effective~\cite{MA07,SG11}. Another simple
gesture is to consciously refrain from speech that is implicitly
judgmental and promotes \emph{defensive social climate}~\cite{BGJ02}.
For instance, the declaration ``smart students who finish early will
receive extra credit'' carries with it the connotation that students
who finish on time or late are not smart. Declarations like these must
be avoided.  It is by combining such simple gestures with the rigors
of teaching CS that I believe we can make CS classrooms more inclusive
and diverse.

\section*{Teaching Experience}

I am fortunate to have gotten the opportunity to serve as Visiting
Assistant Professor at Purdue since August, 2019. For the current
(Fall'19) semester, I am teaching CS307 Software Engineering to
undergraduates in their junior year, which gives me an opportunity to
implement some of my teaching ideals in practice. The course aims to
equip students with various skills that help them organize a software
development project, and deliver reliable software in time and within
budget. The curriculum includes such things as Agile principles,
manual and automated software testing, symbolic modelchecking, and
code review. The course is primarily taken by students who intend to
specialize in software engineering, hence the pedagogical and learning
goals are in perfect alignment. It was therefore not hard for me to
organize my teaching around the principle \emph{utility over
aesthetics; problems over solutions}. For every topic I teach, I give
a sufficiently elaborate demonstration of the utility of the topic in
real-world software development with an emphasis on the problems it
aims to solve. It also helps that students work on a semester-long
software development project, which acts as a familiar common context
for effective demonstration of software engineering concepts. Advanced
software testing and modelchecking algorithms are taught by first
simplifying them to capture their essence (thus prioritizing
\emph{clarity over truth}), and running through the simplified
algorithm with help of several examples (i.e., with sufficient
\emph{rigor}). Due attention is paid to social aspects of teaching
whenever an opportunity presents itself. For instance, when an
advanced testing technique is discussed, I consciously acknowledge the
difficulty in grasping the technique so that the students who did not
understand it do not feel invalidated. Another regular practice is to
use the pronouns ``she'' and ``her'' to address a programmer so as
to counter gender stereotypes prevalent in tech.

At the time of writing this statement, Fall semester is ongoing and
the class is still in session. However, the feedback I got so far,
directly and indirectly, is quite encouraging. At the end of a
particularly engaging recent lecture on concurrency bugs, a student
excitedly walked up to me to let me know that he is thoroughly
enjoying the lectures. An indirect encouragement I receive from
students is in form of the high attendance to my lectures even on days
when there is guaranteed to be no quiz in class. I plan to build on
this encouragement in future, as I look forward to challenging
teaching assignments in Programming Languages, Systems, and Software
Engineering.

\section*{Mentorship}

During my graduate studies at Purdue, I have had the opportunity to
mentor several junior graduate students, few of whom have later became
my research collaborators. It is a matter of pride to me that on at
least four of my research papers, my immediate co-author (2nd or 1st)
is a junior graduate student for whom it was the first ever
publication. I am also comfortable mentoring senior researchers who
are more accomplished than I when they choose to expand into a
research area that is more familiar to me. On one of my papers, my
immediate co-author was a post-doctoral researcher for whom it was the
first ever research paper in programming languages. He has since begun
his own line of exploration, and has become a regular contributor at
PL venues. Beyond one-to-one mentorship, I have also taken initiatives
for collective professional advancement that have stood the test of
time. PurPL, a PL reading group I started in Fall 2013, has since
branched off and matured into a full-fledged research
organization\footnote{ PurPL, as it is known now~\cite{purpl}, is
  largely due to the efforts of Profs. Tiark Rompf, Milind Kulkarni,
  Roopsha Samanta, and Ben Delaware. The reading group continues
  nonetheless.}. I would love to take more such initiatives in the
future for nothing gives me more joy than sharing the excitement of
learning and discovery in Computer Science with my fellow researchers.
