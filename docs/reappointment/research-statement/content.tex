Popular web services, such as Amazon, Google, and Facebook, are supported
by large numbers of \emph{distributed} computing systems that involve the
coordination across thousands of machines to serve millions of requests
every second. Increasingly, these systems are responsible for powering
hospitals, air traffic control systems and traffic signals. Ensuring
fail-safe operation and resilience against adversarial attacks require new
ideas on how to program these massive systems and how to aid programmers in
reasoning about their behaviors. The field of programming languages (PL)
and formal methods (FM) in computer science investigates techniques founded
in symbolic logic and automated reasoning to solve some of these pressing
issues. Formal methods are witnessing increasing adoption at ``high-tech''
organizations to secure mission-critical systems, such as unmanned aerial
vehicles at DARPA and US Air Force~\cite{darpa-hacms,darpa-airforce},
rovers and orbiters at NASA~\cite{nasa-fm}, cryptographic communication
systems at Microsoft~\cite{project-everest}, and planet-scale web services
at Amazon~\cite{aws-formal}. However, widespread adoption of formal methods
remains a challenge due to their steep learning curve, the complexity of
their integration into software development lifecycle, and the
computational hardness of scaling them to real-world systems.
\begin{mdquote}
My research attempts to make foundational advances in programming languages
and formal methods with aim of making provably-safe distributed systems
development accessible to mainstream software developers.
\end{mdquote}
\noindent Bridging the gap between theoretical advances and practical
deployment in industry requires research into more accessible tools,
automated reasoning techniques, and programming abstractions that can
seamlessly integrate with existing development workflows. My work is
motivated by these challenges and aims to make formal methods more
practical and impactful for real-world software systems. My research
program focuses on ($R_1$) the development of formal verification
techniques that can scale to industrial-scale distributed systems; ($R_2$)
the application of ideas from $R_1$ to study problems of adversarial
attacks on distributed blockchains; and ($R_3$) the application of
verification approaches to probabilistic models of physical systems. I will
next summarize each research thrust, highlight significant contributions,
and present their broader impacts and applications. 

\section*{$R_1$ Scaling Formal Verification to Real Systems}

I believe that the key to making formal verification scale to real
distributed systems lies in adopting a programming methodology that is
designed from first principles to be amenable to machine-assisted
reasoning. Just as human languages influence the human thought process,
programming languages influence the design of the computing systems by
prioritizing certain concerns, e.g., efficiency, over others, e.g.,
understandability. Several decades of research into programming languages
has shown that \emph{modularity} and \emph{abstraction} are often the key
to overcoming programming complexity. Intuitively, the programmer should be
able to express the high level modular structure of their software system
while abstracting away the implementation minutia behind well-defined
interfaces. \textbf{The abstractions that lower the cognitive overhead of
comprehension for humans also lower the computational overhead of reasoning
for machines}, leading to scalable formal verification. Unfortunately, the
abstractions typically used to program distributed systems expose low-level
communication primitives and require the explicit reasoning about the
dynamics of the network state in relation to \emph{faults}, such as node
crashes and network partitions. To scale formal verification to real-world
systems implementations, we need novel programming abstractions that
decompose the complexity of fault tolerance and application logic into
loosely-coupled \emph{modules} and \emph{services} amenable to independent
specification and verification.

\begin{mdquote}
Thrust 1 focuses on developing novel programming abstractions that enable
modular specification, verification, and implementation of distributed systems.
\end{mdquote}

The primary motivation for modular decomposition in our case is automated
verification as opposed to efficient implementation. A class of tools,
called Satisfiability-Modulo-Theory (SMT) solvers~\cite{smt-primer}, are
the workhorses of machine-assisted formal verification in practice. To be
amenable to SMT-aided automated reasoning, verification conditions should
be encodable in a a decidable logic, which requires deliberate design of
modules and interfaces. As a part of ongoing work in Thrust 1, we are
developing language-embedded specification and reasoning frameworks that
reduce the manual effort and creative inputs involved in this process. The
broadest impact will be on mainstream application programmers, who may not
have the expertise to build mission-critical distributed systems. Our
research will help programmers overcome the complexity of this task by
making formal verification and machine-assisted reasoning readily
accessible. 

\mypara{Significant Contributions} Our early work in Thrust 1~\cite{mrdt,
snapl19, quark} focused on development of \emph{Mergeable Replicated Data
Types} (MRDTs) -- a high-level distributed programming abstraction that
guarantees fault tolerance by design. We built a distributed runtime for
MRDTs, called \quark~\cite{quark}, that orchestrates distributed
computation in a way that is guaranteed to converge. This is in contrast to
the state-of-the-art RDT abstractions, which require the programmer to
prove the convergence of a computation explicitly~\cite{crdt}. \quark has
allowed us to uncover novel performance tradeoffs in distributed
applications and collect strong experimental evidence demonstrating the
viability of high-level programming abstractions for
\emph{coordination-free} distributed computations. Theoretically, RDTs are
a high-level abstraction of a weakly-consistent replicated state machine
(RSM), denoted $\eventual$ RSM. While $\eventual$ RSMs are often useful,
safety-critical distributed systems need strongly-consistent ($\strong$)
RSMs, whose implementations are known to be notoriously complex. In our
recent work~\cite{lewchenko-oopsla25}, we have shown that $\eventual$ RSMs
can be extended with modules of inter-replica coordination to obtain
$\strong$ RSMs, resulting in a modular approach that drastically reduced
the overhead of safety verification. We implemented this technique in a
verification framework called \superv and used it to automatically verify
an industry-strength implementation of Raft log replication
protocol~\cite{raft}. We believe this is a notable achievement considering
that verification efforts of similar nature required 2-3 orders of
magnitude more manual annotations~\cite{VerdiRaft}. Our results will soon
be presented at SIGPLAN OOPSLA 2025 conference. This line of work was
recognized by Amazon through a research award.

\mypara{Funding and Project Management} The primary source of funding has
thus far been my startup grant. An NSF Small proposal titled \emph{A
Modular Approach to Practical Distributed Systems Verification} is currently
pending review. The proposed budget is \$480,110 to be spent over three
years. The project team includes Mr. Nicholas Lewchenko, a PhD student
advised by me. Mr. Lewchenko will graduate in Spring 2026 and will be
succeeded by a new graduate student yet to be admitted. 

\section*{$R_2$ Generalizing Formal Verification to Adversarial Networks}

Distributed systems now increasingly operate the public internet, where
they are used to implement novel decentralized applications, such as
blockchain cryptocurrencies~\cite{nakamoto-bitcoin08,buterin-eth14} and
federated social networks~\cite{mastodon,bluesky}. A decentralized
application is effectively a \emph{internet-native} distributed system.
unlike conventional distributed systems, which operate inside secure data
centers and private networks, internet-native systems are exposed to
anonymous adversaries who can actively subvert system executions towards
unsafe or unproductive states. Ensuring the safety of distributed systems
and the privacy of their users on adversarial networks requires
synthesizing reasoning and verification techniques from Cryptography and
Distributed Systems.
\begin{mdquote}
  The focus of Thrust 2 is to develop a unified framework that combines cryptographic proofs of security with formal verification of distributed protocols. 
\end{mdquote}
\noindent The goal is to help developers obtain strong guarantees of safety
and privacy in adversarial environments. Like in Thrust 1, we seek modular
abstractions that allow compositional specification and verification of
protocol components, but now with explicit modeling of adversarial actions
and cryptographic primitives. In our ongoing work, we are developing
frameworks that enable automated reasoning about both protocol correctness
and cryptographic soundness, aiming to reduce the manual effort required
for end-to-end verification. This approach will empower developers to build
robust decentralized applications that are resilient to both crash faults
and active attacks, facilitating broader adoption of secure distributed
systems in practice.

\mypara{Significant Contributions} In~\cite{waldo-oopsla23}, we introduced
a novel formal model of cryptography and an associated probablistic
relational verification framework. In formal verification of cryptographic
protocols, a network attacker is modeled in one of the two fundamentally
different ways: (a). a \emph{symbolic} or \emph{Dolev-Yao} attacker defined
in terms of what the attacker can do~\cite{dolev-yao}, and (b). a
\emph{computational} attacker defined in terms of what the attacker can not
do~\cite{blanchet-tr23, bellare-eurocrypt06}. Correspondingly, two standard
models of cryptography -- \emph{Symbolic} and \emph{Computational} -- have
emerged. While the former is simpler and amenable to symbolic reasoning, it
underapproximates attacker capabilities, hence misses several classes of
attacks. Conversely, the latter is more general and precise, but requires
arduous manual proofs. In~\cite{waldo-oopsla23}, we introduced a
probabilistic model of cryptography that combines the strengths of both: it
is more precise than the symbolic model (hence covers more classes of
attacks) and more amenable to automation than the computational model. We
implemented the probabilistic model in an SMT-aided verification framework
called \waldo. \waldo is released publicly with an open source license, and
was used to identify subtle privacy issues in the draft proposal for TLS
Encrypted Client Hello (ECH) extension~\cite{ech-draft}. We believe this is
a significant milestone considering that TLS is the protocol powering
secure HTTP connections for the entire internet, hence any privacy issues
are bound to have adverse consequences. 

\mypara{Funding and Project Management} The primary source of funding has
thus far been my startup grant. My NSF CAREER proposal titled
\emph{Verifying Safety and Privacy of Distributed Systems on Adversarial
Networks} is currently under review. The proposed budget is \$714,519 to be
spent over five years for the activities in Thrust 2. In addition, we have
partnered with Psiphon Inc~\cite{psiphon} to provide inputs to DARPA PM Mr.
Michael Lack on formulating a new program on privacy verification on
internet. The project team includes PhD students Mr. Kirby Linvill and Mr.
Sai Aka, both advised by me. 

\section*{$R_3$ Making Formal Verification Accessible to Non-Experts}

A  significant progress has been made in the formal methods community
towards automating symbolic reasoning for complex systems. However, the
tools and languages developed often require a specialized skill set to
write precise mathematical specifications and proofs, which is often out of
reach for mainstream application programmers. To make formal methods an
accessible and routine part of the software development lifecycle, there is
a need for novel techniques and tools that flatten the learning curve and
lower the barrier to entry. The recent progress in Large Language Models
(LLMs) provides an exciting opportunity to (a). leverage natural languages
as interfaces to formal verification tools, and (b). use Generative AI to
guide the computationally intensive proof search process performed by every
formal verification tool. There are also notable new inventions in
programming language techniques, such as the Incorrectness
Logic~\cite{ohearn-popl19} and Goal-directed abstract
interpretation~\cite{historia-oopsla23}, which require minimal user inputs
to drive formal verification. In Thrust 3, our goal is to leverage these
developments to drastically improve the usability of formal methods,
particularly in the context of distributed systems. 
\begin{mdquote}
Thrust 3 focuses on developing symbolic and neuro-symbolic reasoning
techniques to automate formal verification for distributed systems. 
\end{mdquote}

\mypara{Significant Contributions} In~\cite{lewchenko-oopsla26}, we
proposed a novel partial-function semantics for higher-order functional
programs that makes them amenable to decidable encoding in SMT solvers.
Partial functions underapproximate total functions, which makes formal
verification unsound in general. Our key contribution is a set of
sufficient conditions under which the underapproximation is actually sound.
We developed a language extension to Rust, called $\lepr$, that lets
non-expert programmers use SMT solvers to automatically verify deep
semantic properties of higher-order functional programs. Another notable
contribution is the \dissprove verification
framework~\cite{fontenot-dissprove} for distributed protocols that
leverages a novel goal-directed backwards analysis we introduced in our
earlier work~\cite{meier-oopsla23} to eliminate the need for inductive
invariants (for a class of protocols). Identifying the right inductive
invariants is often hardest step in formal verification of distributed
protocols~\cite{ivy2016}. By short-circuiting this process, \dissprove
makes it feasible for non-experts to build provably-safe distributed
systems.

\mypara{Ongoing Work} In the above, we have only considered systems with
discrete dynamics. Our ongoing work extends formal verification to systems
with continuous dynamics where state transitions are defined in terms of
ordinary differential equations (ODEs). The complex structure of the ODEs
often makes it expensive to perform computations such as simulations and
probabilistic inference. In our ongoing work, we are building a DSL for
probabilistic programming with ODEs coupled with a verified optimizing
compiler. The compiler orchestrates a series of semantics-preserving
transformations, such as latent variable elimination, to automatically
transform ODEs into a form amenable to efficient inference. This allows
domain experts to focus on specifying system dynamics without concerning
themselves with efficient implementations. In another thread of ongoing
work, we showed that purely neural approaches, including the
state-of-the-art LLMs, perform poorly on program synthesis
tasks~\cite{roberson-arxiv2024}. To improve the effectiveness of LLM-driven
program synthesis, we designed a semantics-constrained decoding technique
that uses conventional symbolic techniques to prune unproductive inference
paths. The early results are promising and we are working on developing a
stand-alone tool to apply our insights at scale. 

\mypara{Funding and Project Management} The verified ODE transformation
work is funded by the NSF Formal Methods in the Field (FMitF) program from
09/01/2024 to 08/31/2028 through award number: 2422136, amount: \$875,000,
titled: \emph{FMitF: Track I: Verified Probabilistic Programming for Hybrid
Systems}. An Amazon Research Awards (ARA) proposal titled \emph{Program
Synthesis with Syntax-Aligned Language Models} is currently under review.
The project team includes Mr. Oscar Bender-Stone, an undergraduate student
on Discovery Learning Apprenticeship (DLA) program, along with the PhD
students Mr. Christian Fontenot, Mr. Nicholas Lewchenko, Mr. Kirby Linvill,
and Ms. Manasvi Parekh. 



