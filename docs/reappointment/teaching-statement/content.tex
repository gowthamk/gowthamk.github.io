%\begin{mdquote}
%``Tell me and I forget, teach me and I may remember, involve me and I learn.'' --- Benjamin Franklin
%\end{mdquote}

Programming language is the primary medium of
communication between humans and machines. A well-designed programming
language must serve as a natural medium of expressing human ideas about
computation and should be able translate those ideas to efficient
machine-executable code while preserving the \emph{semantics}. To be
considered competent programmers and computer scientists, especially in the
age of Large Language Models (LLMs), students should have the skill to
quickly pick up new programming languages along with the ability to build
new languages if needed. This is only possible if the students understand
the scientific principles underpinning programming languages rather than
merely knowing how to use an array of programming languages. I see this
analogous to studying linguistics versus merely learning a new language on
DuoLingo: the former requires a scientific approach whereas the latter can
be done recreationally. My approach to teaching programming languages (PL)
courses is informed by this observation. 
\begin{mdquote} 
My teaching mission is to make rigorous programming languages education
widely accessible. 
\end{mdquote}
\noindent For the PL courses I teach, I designed the curriculum (sometimes
from the ground-up) and tailored my instruction to familiarize the
scientific method to students inasmuch as it becomes their second nature.
Considering that PL courses of this nature are only taught at top-tier
research universities, I publish my course materials online -- on dedicated
course-specific websites -- to make them widely available to anyone who is
willing to learn.

\section*{Educational Contributions}

Since joining CU Boulder, I have participated in the educational mission of both the Programming Languages and Verification (PLV) and the Systems groups. I have taught graduate courses such as:
\begin{itemize}
    \item \textbf{Fundamentals of Programming Languages} (CSCI 5535/ECEN 5533; Fall 2021, Fall 2024)
    \item \textbf{Advanced Functional Programming} (CSCI 7000; Spring 2023)
    \item \textbf{Distributed Systems Verification} (CSCI 7000; Spring 2021, Spring 2024)
\end{itemize}
At the undergraduate level, I have taught \textbf{Programming Languages}
(CSCI 3155) to hundreds of students across multiple semesters (Spring 2021,
Spring 2022, Fall 2023, Spring 2025). In CSCI 3155, students explore topics
such as functional programming, operational semantics, interpreters, type
systems, continuations, object-oriented programming, and garbage
collection. In my graduate courses, topics range from formal semantics,
type theory, and program verification to advanced functional programming
paradigms, distributed algorithms, and verification techniques. I
re-designed the course curriculum of CSCI 5535 to make extensive use of
mechanical proof assistants, such as Coq and Lean, which have now become
popular in the mathematics research community owing to their rigor.
Detailed schedules and materials for all my courses are publicly available:
\begin{itemize}
    \item CSCI 5535: \url{https://csci5535.cs.colorado.edu/f24/}
    \item CSCI 7000 Advanced Functional Programming: \url{https://gowthamk.github.io/csci7000_pfp_s23/}
    \item CSCI 7000 Distributed Systems Verification: \url{https://gowthamk.github.io/csci7000_s21/}
\end{itemize}

The instructional materials listed on the course websites were largely
developed by myself (others' contributions have been duly credited). These
websites not only provide students with lecture notes, assignments, and
code but also serve as open resources for the broader academic community.

\section*{Course Design and Delivery}

My classes are highly interactive: I use literate programming-style Jupyter
notebooks to seamlessly switch between prose and programming. When I teach
programming, I make it a point to live-code the examples in class while
inviting inputs from students. This method of collaborative problem solving
inculcates a sense of community among students and lowers the barrier to
class participation. I also make heavy use of iPad in the class in lieu of
whiteboard to help students watching classrooom capture videos follow along
easily. In addition, I integrate project-based learning wherever
possible—asking students to build interpreters, experiment with language
features, or verify non-trivial protocols. These projects give students the
dual satisfaction of mastering theory while producing something functional
and relevant. As described earlier, I maintain public course websites for
my graduate-level courses, which play a crucial role in my teaching. Beyond
providing transparency and accessibility, they allow students to revisit
materials at their own pace and serve as a portfolio of my evolving
pedagogical approach.

\section*{Making Utility Visible}

My teaching philosophy is grounded in the principle of demonstrating
utility: students truly learn only when they can see a skill's practical
value in their own lives and future work. When learners recognize how a
concept connects to problems they care about, the motivation to engage
deeply follows naturally. In my classrooms, I strive to bridge theory with
live application. When introducing an abstract concept—be it continuations
in functional programming or model checking in distributed systems—I begin
by situating the idea in a context the students can relate to. This could
mean:
\begin{itemize}
    \item Demonstrating how type systems prevent security vulnerabilities in real-world software.
    \item Connecting garbage collection algorithms to memory management issues in industry-scale systems.
    \item Linking verification techniques to the reliability demands of distributed systems like blockchain or NASA flight software.
\end{itemize}
By framing concepts in terms of tangible utility, students not only engage but also begin to transfer their learning to new, unfamiliar domains.

\section*{Mentorship}
Mentoring is where my teaching philosophy extends most naturally into research. Since joining CU, I have advised 4 Ph.D. students, 3 M.S. students, and 7 undergraduates. All 7 of my major publications at CU have been co-authored with CU students. 

Highlights of my mentoring include:
\begin{itemize}
    \item My senior Ph.D. student, Nicholas Lewchenko, is on track to graduate in Fall 2025.
    \item Undergraduate thesis student Pranav Subramanian defended his thesis and joined NASA Goddard Space Flight Center as a Software Engineer.
    \item Two of my undergraduate mentees came from front range community colleges through the SPUR internship program.
    \item Current undergraduates Noah Schwartz and Oscar Bender-Stone are applying to Ph.D. programs this year.
\end{itemize}

As part of the CUPLV group, my students benefit from a vibrant research community. I help organize a weekly student-led seminar and twice-weekly status meetings, in addition to one-on-one weekly advising sessions. These structures foster collaboration, peer learning, and sustained research momentum. 

Mentoring, to me, is a reciprocal process: my students bring fresh perspectives that challenge me to re-examine familiar problems, while I provide the guidance, focus, and encouragement needed to turn promising ideas into polished contributions.

\section*{Closing Thoughts}
Whether in the classroom or the research lab, my aim is to spark curiosity
by showing students the scientific concepts underlying the computational
phenomena. When they can see how an idea might shape their own future
projects or careers, the drive to master it comes naturally—and that, in my
view, is when real learning happens.
