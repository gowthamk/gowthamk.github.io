% LaTeX file for resume 
% This file uses the resume document class (res.cls)

\documentclass[margin,line]{res} 
% the margin option causes section titles to appear to the left of body text 
\textwidth=5.2in % increase textwidth to get smaller right margin
%\usepackage{helvetica} % uses helvetica postscript font (download helvetica.sty)
%\usepackage{newcent}   % uses new century schoolbook postscript font 

\begin{document} 
 
\name{Gowtham Kaki} 
 
%\fontsize{9pt}{1} \selectfont
\address{http://gowthamk.github.io \hspace{1.25in} gkaki@cs.purdue.edu
          \hspace{1.25in} (201) 417 1775 }

 
\begin{resume} 
 
\section{Interests} 
Program Logics, Type Systems and Static Analyses.

\section{Education} 
{\bf Ph.D., Computer Science}. \hfill 2012 - present\\
Advised by Prof. Suresh Jagannathan.\\
Purdue University, West Lafayette, IN.\\
Current GPA: 3.77/4.0 \\
\\
{\bf B.E.(Hons)., Computer Science}. \hfill July, 2009\\
BITS, Pilani, India.\\
GPA: 8.68/10.0
 

\section{Research}
 {\bf Relational Framework for Higher-order Shape Analysis} \\
 (With Suresh Jagannathan) Devised a verification framework based on
 decidable relational logics to automatically reason about
 catamorphisms in ML-like languages. The framework manifests as a
 dependent type system for ML and scales to higher-order programs by
 assigning "very general" (although not principal) types to
 higher-order functions. Our experience so far suggests that the
 method is quite useful to automatically verify partial correctness
 properties of compiler transformations. The current (partial)
 implementation of the framework is available online. Work-in-progress
 presentations have been delivered at 
 \begin{itemize} \itemsep -2pt  % reduce space
 \item Higher-Order Program Analysis (HOPA) Workshop, New Orleans,
 28th-29th June, 2013.
 \item Mid-west Verification Day (MVD), Chicago, 20th-21st September,
 2013.
 \end{itemize}

{\bf A Novel Adaptive Scheduling Algorithm for Computational Grids}\\
(With S. Bansal, and Chittaranjan Hota) Devised a de-centralized
dynamic load balancing algorithm for efficient task scheduling in
computational grids.
 \begin{itemize} \itemsep -2pt  % reduce space
 \item IEEE conference on Internet Multimedia Systems Architecture and
 Applications (IMSAA), Bangalore, India, 2009.
 \end{itemize}

\section{Professional \\ Experience}
{\bf Software Engineer, Yahoo SDC, Bangalore, India} \hfill August, 2009 -
July, 2011\\
Frontend engineering for Yahoo content platforms group. Developed AJAX
and php tools for querying, processing and presenting
loosely-structured data from various content grids inside yahoo. The
tools were used by Yahoo's content curators.

{\bf Engineering intern, Qualcomm, Hyderabad, India} \hfill January,
2009 - June, 2009\\
QA Engineering for Application-specific integrated circuit (ASIC) -
User interface module (UIM) group. Developed tools to test low-level
mobile network code.

\section{Academics}
{\bf Relevant Graduate Coursework }\\
\begin{itemize}
\item CS565 Programming Languages (analogoue of UPenn's Software
Foundations with Coq), CS510 Software Engineering, EE570 Artificial
Intelligence, and CS590 Current topics in Theoretical Computer
Science.
\item Attended Oregon Programming Languages Summer School (OPLSS'13)
at Eugene, OR.
\end{itemize}

{\bf Teaching Assistanship}\\
\begin{itemize}
\item CS240 C Programming. Fall 2012.
\item CS565 Programming Languages. Spring, 2013.
\end{itemize}

\section{Programming \\ Languages}
Standard ML, Ocaml, Scheme, C, Java, and Haskell (pure Haskell, to be
precise. I am not yet comfortable with monadic types.).

\section {Service}
Coordinating weekly sessions of Purdue PL (PurPL) reading group.
Notes/Slides for some sessions when I led the discussion are available
on my web page.\\
Purdue CS Graduate Student Board (GSB) office member, Fall 2011 and
Spring 2012.\\
Secretary (junior year), and office member (freshman and sophomore
years) of Computer Science Association (CSA) BITS, Pilani. We
organized our techfest (APOGEE) in 2008. \\

\section{References}
Prof. Suresh Jagannathan (suresh@cs.purdue.edu).\\
Prof. Chittaranjan Hota (hota@hyderabad.bits-pilani.ac.in)\\
Other references will be available on request.
\end{resume} 
\end{document} 



