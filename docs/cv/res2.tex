% LaTeX file for resume 
% This file uses the resume document class (res.cls)

\documentclass[margin,line]{res} 
% the margin option causes section titles to appear to the left of body text 
\textwidth=5.2in % increase textwidth to get smaller right margin
%\usepackage{helvetica} % uses helvetica postscript font (download helvetica.sty)
%\usepackage{newcent}   % uses new century schoolbook postscript font 
\usepackage{hyperref}

\begin{document} 
 
\name{Gowtham Kaki} 
 
%\fontsize{9pt}{1} \selectfont
\address{http://gowthamk.github.io \hspace{1.25in} gkaki@purdue.edu
          \hspace{1.25in} (201) 417 1775 }

 
\begin{resume} 
 
\section{Interests} 
Program Logics, Type Systems and Static Analyses.

\section{Education} 
{\bf Ph.D., Computer Science}. \hfill 2012 - present\\
Advised by Prof. Suresh Jagannathan.\\
Purdue University, West Lafayette, IN.\\
Current GPA: 3.64/4.0 \\
\\
{\bf B.E.(Hons)., Computer Science}. \hfill July, 2009\\
BITS, Pilani, India.\\
GPA: 8.68/10.0
 

\section{Research}
 {\bf Declarative Programming over Eventually Consistent Data Stores} \\
 (With Sivaramakrishnan KC., and Suresh Jagannathan) Devised a
 logic-based language framework to express high-level consistency
 requirements of NoSQL applications. The framework delivers
 application programmers from having to manually tune consistency
 levels of NoSQL stores, such as Cassandra, which are often tied to
 the implementation nuances of these stores. Further, the framework
 comes with an extensive support for scalable transactions, which is
 needed to build real-world NoSQL applications. The implementation of
 the framework (called {\sc Quelea}) is available online at\\
 \url{https://github.com/kayceesrk/Quelea}.
 \begin{itemize}
 \item Submitted to ACM SIGPLAN Conference on Programming Languages
 Design and Implementation (PLDI), 2015. Draft paper avalilable at
 \url{http://gowthamk.github.io/docs/quelea.pdf}.
 \end{itemize}


 {\bf Relational Framework for Higher-order Shape Analysis} \\
 (With Suresh Jagannathan) Devised a specification language based on a
 decidable relational logic to automatically reason about structural
 properties in ML-like languages. Our specification language makes it
 possible to capture rich semantics of data structure transformations
 as logical formulas, and subsequently use them for either
 verification or proving program equivalence. The verification
 framework for our language (called {\sc Catalyst}) has been
 implemented, and is available for experimentation at
 \url{http://tycon.github.io/catalyst/}.
 \begin{itemize} \itemsep -2pt  % reduce space
 \item Published in the proceedings of the International Conference on
 Functional Programming (ICFP), Gothenburg, 1st-6th September, 2014.
 \item Presented at the Higher-Order Program Analysis (HOPA) Workshop,
 New Orleans, 28th-29th June, 2013.
 \item Presented at the Mid-west Verification Day (MVD), Chicago,
 20th-21st September, 2013.
 \item Ongoing work on extending the framework with specification
 inference is presented at the Mid-west Verification Day (MVD),
 Columbia, MO, 2nd-3rd October, 2014.
 \end{itemize}

{\bf A Novel Adaptive Scheduling Algorithm for Computational Grids}\\
(With S. Bansal, and Chittaranjan Hota) Devised a de-centralized
dynamic load balancing algorithm for efficient task scheduling in
computational grids.
 \begin{itemize} \itemsep -2pt  % reduce space
 \item Published in the proceedings of IEEE conference on Internet
 Multimedia Systems Architecture and Applications (IMSAA), Bangalore,
 India, 2009.
 \end{itemize}

\section{Professional \\ Experience}

{\bf Research Intern, Microsoft Research India, Bangalore} \hfill May - August, 2014\\
(With G Ramalingam, K Vaswani, and D Vytiniotis) Built a region type
system for Microsoft Dryad programs to ensure memory safety in
presence of programmer-managed memory regions. Also developed a type
inference algorithm that will automatically infer region types even in
presence of higher-order functions. A paper describing the type system
will be available from April, 2015.

{\bf Software Engineer, Yahoo SDC, Bangalore, India} \hfill August, 2009 -
July, 2011\\
Frontend engineering for Yahoo content platforms group. Developed AJAX
and php tools for querying, processing and presenting
loosely-structured data from various content grids inside yahoo. The
tools were used by Yahoo's content curators.

{\bf Engineering intern, Qualcomm, Hyderabad, India} \hfill January - June, 2009\\
QA Engineering for Application-specific integrated circuit (ASIC) -
User interface module (UIM) group. Developed tools to test low-level
mobile network code.\\

\section{Academics}
{\bf Relevant Graduate Coursework (With Grades)}\\
\begin{itemize}
\item Programming Languages (A), Software Engineering (B),
Metaprogramming and Program Generation (A+), Distributed Systems (A),
Parallel Computing (A), and Current topics in Theoretical Computer
Science (A-).  
\item Attended Oregon Programming Languages Summer School (OPLSS'13)
at Eugene, OR.
\end{itemize}

{\bf Courses Handled as a Teaching Assistant}\\
\begin{itemize}
\item CS240 C Programming. Fall 2012.
\item CS565 Programming Languages. Spring, 2013.
\end{itemize}

\section{Grants and Scholarships}
\begin{itemize}
\item Received ACM SIGPLAN PAC travel grant for ICFP 2014.
\item Received ACM SIGPLAN PLMW scholarship for POPL 2014.
\item Received institute merit-cum-need scholarship during all 
semesters of my undergraduate education at BITS, Pilani, India.
\end{itemize}

\section {Professional Service}
\begin{itemize}
\item Coordinating weekly sessions of Purdue PL (PurPL) reading group.
Notes/Slides for some sessions when I led the discussion are available
on my web page.
\item Purdue CS Graduate Student Board (GSB) office member, Fall 2011 and
Spring 2012.
\item Secretary (junior year), and office member (freshman and sophomore
years) of Computer Science Association (CSA) BITS, Pilani. We
organized our techfest (APOGEE) in 2008. 
\end{itemize}

\section{References}
Prof. Suresh Jagannathan (Purdue University).\\
Dr. Ganesan Ramalingam (Microsoft Research). \\
Other references will be available on request.
\end{resume} 
\end{document} 



