% LaTeX file for resume 
% This file uses the resume document class (res.cls)

\documentclass[margin,line]{res} 
% the margin option causes section titles to appear to the left of body text 
\textwidth=5.2in % increase textwidth to get smaller right margin
%\usepackage{helvetica} % uses helvetica postscript font (download helvetica.sty)
%\usepackage{newcent}   % uses new century schoolbook postscript font 
\usepackage{hyperref}

\newcommand{\code}[1]{\,{\tt #1}\,}
\newcommand{\C}[1]{\code{#1}}

\begin{document} 
 
\name{Gowtham Kaki} 
 
%\fontsize{9pt}{1} \selectfont
\address{http://gowthamk.github.io \hspace{1.25in} gkaki@purdue.edu
          \hspace{1.25in} (201) 417 1775 }

 
\begin{resume} 
 
\section{Interests} 
Program Logics, Type Systems and Static Analyses.

\section{Education} 
{\bf Ph.D., Computer Science}. \hfill 2012 - present\\
Advised by Prof. Suresh Jagannathan.\\
Purdue University, West Lafayette, IN.\\
Current GPA: 3.77/4.0 \\
\\
{\bf B.E.(Hons)., Computer Science}. \hfill July, 2009\\
BITS, Pilani, India.\\
GPA: 8.68/10.0
 

\section{Research}

 {\bf Broom : Safe Region-based Memory Management for Dataflow
 Systems}\\ (With G Ramalingam, K Vaswani, and D Vytiniotis)
 Distributed data processing engines (e.g: MapReduce) rely on
 general-purpose language runtimes to manage the execution of dataflow
 operators like \C{Join}. We observe that the memory behavior of such
 operators exhibit certain characteristics, which render
 general-purpose garbage collection a particularly ineffective memory
 management strategy. To address this problem, we propose {\sc Broom},
 dialect of C\# with programmer-managed memory regions. 
 %Regions are first-class values in
 %{\sc Broom}: they can be stored in data structures, and can be passed
 %as arguments to methods. Despite this expressivity, 
 {\sc Broom} empowers dataflow system builders to implement efficient
 region-based memory management strategies without incurring
 significant cost: it's region type system guarantees safety of all
 dereferences that are otherwise safe in presence of GC, while it's
 region type inference obviates any need for programmer-provided
 region type annotations. Experiments with real-world dataflow
 operators demonstrate upto 59\% performance improvement due to safe
 region-based memory managegement afforded via {\sc Broom}. An OCaml
 implementation of {\sc Broom}'s region type system and type inference
 can be obtained from \url{https://github.com/gowthamk/broomc}.
 \begin{itemize}
 \item Draft paper available at
 \url{http://gowthamk.github.io/docs/broom.pdf}
 \item Work-in-progress presentations at Microsoft Research India,
 Bangalore in August, 2014 and July, 2015. 
 \end{itemize}

 {\bf Declarative Programming over Eventually Consistent Data Stores} \\
 (With Sivaramakrishnan KC., and Suresh Jagannathan) Devised a
 logic-based language framework to express high-level consistency
 requirements of NoSQL applications. The framework delivers
 application programmers from having to manually tune consistency
 levels of NoSQL stores, such as Cassandra, which are often tied to
 the implementation nuances of these stores. The framework also comes
 with an extensive support for scalable transactions. The
 implementation of the framework (called {\sc Quelea}) is available
 online at \url{https://github.com/kayceesrk/Quelea}.
 \begin{itemize}
 \item Published in the proceedings of ACM SIGPLAN Conference on
 Programming Languages Design and Implementation (PLDI'15)
 \item Presented at PLDI'15 at Portland, June 13-17, 2015.
 \item Presented at RF Seminar Series, Microsoft Research India,
 Bangalore in July, 2015.  
 \end{itemize}


 {\bf Relational Framework for Higher-order Shape Analysis} \\
 (With Suresh Jagannathan) Devised a specification language based on a
 decidable relational logic to automatically reason about structural
 properties in ML-like languages. Our specification language makes it
 possible to capture rich semantics of data structure transformations
 as logical formulas, and subsequently use them for either
 verification or proving program equivalence. The verification
 framework for our language (called {\sc Catalyst}) has been
 implemented, and is available for experimentation at
 \url{http://tycon.github.io/catalyst/}.
 \begin{itemize} \itemsep -2pt  % reduce space
 \item Published in the proceedings of the International Conference on
 Functional Programming (ICFP'14).
 \item Presented at ICFP'14, Gothenburg, 1st-6th September, 2014.
 \item Work-in-progress presentation at the Mid-west Verification Day
 (MVD), Chicago, 20th-21st September, 2013.
 \item Work-in-progress presentation on extending the framework with
 specification inference at the Mid-west Verification Day (MVD),
 Columbia, MO, 2nd-3rd October, 2014.
 \end{itemize}

{\bf A Novel Adaptive Scheduling Algorithm for Computational Grids}\\
(With S. Bansal, and Chittaranjan Hota) Devised a de-centralized
dynamic load balancing algorithm for efficient task scheduling in
computational grids.
 \begin{itemize} \itemsep -2pt  % reduce space
 \item Published in the proceedings of IEEE conference on Internet
 Multimedia Systems Architecture and Applications (IMSAA), Bangalore,
 India, 2009.
 \end{itemize}

\section{Professional \\ Experience}

{\bf Research Intern, Microsoft Research India, Bangalore} \\
(May - August, 2014 \& July-August, 2015) 
Built a region type system and region type inference to ensure the
safety of dataflow programs that rely on programmer-managed memory
regions, instead of garbage collection for memory management.

{\bf Software Engineer, Yahoo SDC, Bangalore, India} \hfill August, 2009 -
July, 2011\\
Frontend engineering for Yahoo content platforms group. Developed AJAX
and php tools for querying, processing and presenting
loosely-structured data from various content grids inside yahoo. The
tools were used by Yahoo's content curators.

{\bf Engineering intern, Qualcomm, Hyderabad, India} \hfill January - June, 2009\\
QA Engineering for Application-specific integrated circuit (ASIC) -
User interface module (UIM) group. Developed tools to test low-level
mobile network code.\\

\section{Academics}
{\bf Relevant Graduate Coursework (With Grades)}\\
\begin{itemize}
\item Design \& Analysis of Algorithms (A), Programming Languages (A),
Software Engineering (B), Metaprogramming and Program Generation (A+),
Distributed Systems (A), Parallel Computing (A), and Current topics in
Theoretical Computer Science (A-).  
\item Attended Oregon Programming Languages Summer School (OPLSS'13)
at Eugene, OR.
\end{itemize}

{\bf Courses Handled as a Teaching Assistant}\\
\begin{itemize}
\item CS240 C Programming. Fall 2012.
\item CS565 Programming Languages. Spring, 2013.
\end{itemize}

\section{Grants and Scholarships}
\begin{itemize}
\item NSF Travel grant for PLDI 2014
\item ACM SIGPLAN PAC travel grant for ICFP 2014.
\item ACM SIGPLAN PLMW scholarship for POPL 2014.
\item Institute merit-cum-need scholarship during all 
semesters of my undergraduate education at BITS, Pilani, India.
\end{itemize}

\section {Professional Service}
\begin{itemize}
\item Coordinating weekly sessions of Purdue PL (PurPL) reading group.
Notes/Slides for some sessions when I led the discussion are available
on my web page.
\item Purdue CS Graduate Student Board (GSB) office member, Fall 2011 and
Spring 2012.
\item Secretary (junior year), and office member (freshman and sophomore
years) of Computer Science Association (CSA) BITS, Pilani. We
organized our techfest (APOGEE) in 2008. 
\end{itemize}

\section{References}
Prof. Suresh Jagannathan (Purdue University).\\
Dr. Ganesan Ramalingam (Microsoft Research). \\
Other references will be available on request.
\end{resume} 
\end{document} 



