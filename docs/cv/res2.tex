% LaTeX file for resume 
% This file uses the resume document class (res.cls)

\documentclass[margin,line]{res} 
% the margin option causes section titles to appear to the left of body text 
\textwidth=5.2in % increase textwidth to get smaller right margin
%\usepackage{helvetica} % uses helvetica postscript font (download helvetica.sty)
%\usepackage{newcent}   % uses new century schoolbook postscript font 
\usepackage{hyperref}

\begin{document} 
 
\name{Gowtham Kaki} 
 
%\fontsize{9pt}{1} \selectfont
\address{http://gowthamk.github.io \hspace{1.25in} gkaki@purdue.edu
          \hspace{1.25in} (201) 417 1775 }

 
\begin{resume} 
 
\section{Interests} 
Program Logics, Type Systems and Static Analyses.

\section{Education} 
{\bf Ph.D., Computer Science}. \hfill 2012 - present\\
Advised by Prof. Suresh Jagannathan.\\
Purdue University, West Lafayette, IN.\\
Current GPA: 3.64/4.0 \\
\\
{\bf B.E.(Hons)., Computer Science}. \hfill July, 2009\\
BITS, Pilani, India.\\
GPA: 8.68/10.0
 

\section{Research}
 {\bf Declarative Programming over Eventually Consistent Datastores} \\
 (With Sivaramakrishnan KC., and Suresh Jagannathan) Devised a
 logic-based language framework to express high-level consistency
 requirements of NoSQL applications. The framework delivers
 application programmers from having to manually tune consistency
 levels of NoSQL stores, such as Cassandra, which are often tied to
 the implementation nuances of these stores. Further, the framework
 comes with an extensive support for scalable transactions, which is
 needed to build real-world NoSQL applications. The implementation of
 the framework (called {\sc Quelea}) is available online at\\
 \url{https://github.com/kayceesrk/Quelea}.
 \begin{itemize}
 \item Submitted to ACM SIGPLAN Conference on Programming Languages
 Design and Implementation (PLDI), 2015. Draft paper avalilable at
 \url{http://gowthamk.github.io/docs/quelea.pdf}.
 \end{itemize}


 {\bf Relational Framework for Higher-order Shape Analysis} \\
 (With Suresh Jagannathan) Devised a verification framework based on
 decidable relational logics to automatically reason about structural
 properties in ML-like languages. The framework manifests as a
 dependent type system for ML and scales to higher-order programs by
 assigning "very general" types to higher-order functions. Our
 experience so far suggests that the method is quite useful to
 automatically verify partial correctness properties of compiler
 transformations. A web interface to experiment with our
 implementation of the framework (called {\sc Catalyst}) is available
 online at \url{http://tycon.github.io/catalyst/}.
 \begin{itemize} \itemsep -2pt  % reduce space
 \item International Conference on Functional Programming (ICFP),
 Gothenburg, 1st-6th September, 2014.
 \item Higher-Order Program Analysis (HOPA) Workshop, New Orleans,
 28th-29th June, 2013.
 \item Mid-west Verification Day (MVD), Chicago, 20th-21st September,
 2013.
 \end{itemize}

{\bf A Novel Adaptive Scheduling Algorithm for Computational Grids}\\
(With S. Bansal, and Chittaranjan Hota) Devised a de-centralized
dynamic load balancing algorithm for efficient task scheduling in
computational grids.
 \begin{itemize} \itemsep -2pt  % reduce space
 \item IEEE conference on Internet Multimedia Systems Architecture and
 Applications (IMSAA), Bangalore, India, 2009.
 \end{itemize}

\section{Professional \\ Experience}

{\bf Research Intern, Microsoft Research India, Bangalore} \hfill May - August, 2014\\
(With G Ramalingam, K Vaswani, and D Vytiniotis) Built a region type
system for Microsoft Dryad programs to ensure memory safety in
presence of programmer-managed memory regions. Also developed a type
inference algorithm that will automatically infer region types even in
presence of higher-order functions. A paper describing the type system
will be available from April, 2015.

{\bf Software Engineer, Yahoo SDC, Bangalore, India} \hfill August, 2009 -
July, 2011\\
Frontend engineering for Yahoo content platforms group. Developed AJAX
and php tools for querying, processing and presenting
loosely-structured data from various content grids inside yahoo. The
tools were used by Yahoo's content curators.

{\bf Engineering intern, Qualcomm, Hyderabad, India} \hfill January - June, 2009\\
QA Engineering for Application-specific integrated circuit (ASIC) -
User interface module (UIM) group. Developed tools to test low-level
mobile network code.\\

\section{Academics}
{\bf Relevant Graduate Coursework }\\
\begin{itemize}
\item Programming Languages (Software Foundations with Coq), Software
Engineering, Artificial Intelligence, Metaprogramming and Program
Generation, and Current topics in Theoretical Computer Science.
\item Attended Oregon Programming Languages Summer School (OPLSS'13)
at Eugene, OR.
\end{itemize}

{\bf Courses Handled as a Teaching Assistant}\\
\begin{itemize}
\item CS240 C Programming. Fall 2012.
\item CS565 Programming Languages. Spring, 2013.
\end{itemize}

\section{Grants and Scholarships}
\begin{itemize}
\item Received ACM SIGPLAN PAC travel grant for ICFP 2014.
\item Received ACM SIGPLAN PLMW scholarship for POPL 2014.
\item Received institute merit-cum-need scholarship during all 
semesters of my undergraduate education at BITS, Pilani, India.
\end{itemize}

\section {Professional Service}
\begin{itemize}
\item Coordinating weekly sessions of Purdue PL (PurPL) reading group.
Notes/Slides for some sessions when I led the discussion are available
on my web page.
\item Purdue CS Graduate Student Board (GSB) office member, Fall 2011 and
Spring 2012.
\item Secretary (junior year), and office member (freshman and sophomore
years) of Computer Science Association (CSA) BITS, Pilani. We
organized our techfest (APOGEE) in 2008. 
\end{itemize}

\section{References}
Prof. Suresh Jagannathan (Purdue University).\\
Dr. Ganesan Ramalingam (Microsoft Research). \\
Other references will be available on request.
\end{resume} 
\end{document} 



