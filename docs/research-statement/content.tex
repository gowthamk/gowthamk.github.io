Computing has come a long way since Von Neumann introduced his
stored-program model of a computer in 1945. Von Neumann's architecture
is fundamentally \emph{compute-centric} -- a Central Processing Unit
(CPU) performs all the computation by periodically fetching data and
programs from a passive main memory or a secondary storage. Implicit
in this model is the assumption that it is possible to tread a
computation's data in its entirety through a ``central'' processor to
a compute a single value that is the canonical result of a
computation. This view of computing stands in stark contrast to the
reality of modern applications, which are fundamentally
\emph{data-centric} as opposed to compute-centric -- they operate on
large amounts of data organized into multiple data banks, e.g.,
\emph{databases}, \emph{shards}, and \emph{replicas}, and perform
computations that are essentially distributed in nature to compute
results for which no canonical answer exists. Despite such fundamental
differences in the nature of computing, applications today are
nonetheless programmed using the same language abstractions and
programming techniques that trace their origins to the Von Neumann
model. The resultant \emph{impedance mismatch} between the computation
performed and the programming model employed has been a consistent
source of negative impact on the security, reliability, and
performance of data-centric applications. For instance, coercing
database transactions, which are concurrent by their very nature, into
Von Neumann model of computing, which emphasizes an orderly
execution of instructions, has resulted in an impedance mismatch that
effectively exposed several database-backed commercial applications
to concurrency-related attacks~\cite{acidrain}. One such attack has
been particularly notorious for it has led to the theft of over
half-a-million dollars, and subsequent closure of an otherwise
well-performing Bitcoin exchange named Flexcoin~\cite{poloniexbug}.
More such examples abound in literature and in popular press.

% requires explicit enforcement of a concurrency control mechanism
% called Serializability. By guaranteeing equivalence to an orderly
% execution of transactions, Serializability lets application
% developers stay within the familiar confines of Von Neumann model,
% and has therefore gained universal acceptance as the de facto
% standard of database systems correctness. In practice however,
% database systems rarely enforce serializability owing to its
% expensive nature, choosing instead to employ lenient forms of
% concurrency control 

\section*{My Research}

As evident from above, there is a need for novel computational models,
language abstractions and programming techniques that better suit the
nature of modern data-centric applications. Developing such
programming infrastructure is indeed one of the focal points of my
research, in particular my most recent work on distributed programming
models and replicated data structures~\cite{oopsla19,snapl19,thesis}.
The more pressing need however is for automated reasoning techniques,
program analyses, and compilation tools that work with the current
generation of data-centric applications, helping them meet the ever so
strict goals of scalability and availability \emph{without}
compromising security and robustness. Developing such automated
reasoning infrastructure is another focal point of my research, and a
unifying theme behind most of my
work~\cite{popl18,oopsla18,ecoop18,icfp14,pldi15}. \textbf{At a high level, my
research falls at the intersection of programming languages, software
engineering, databases, and distributed systems. In particular, I aim
to solve the most pressing problems in the latter two domains by
developing novel approaches that build on top of the conventional
wisdom and technical know-how of the former two}. The relevance
of my work to the theory and practice of contemporary data-centric
computing has been recognized through a generous research fellowship
from Google. In what follows, I describe my research at a high-level
categorizing it into three broad themes, and conclude by briefly
describing my ideas for future research.

\section*{Theoretical Foundations of Data Storage and Processing
Systems}

The theoretical and mental models of data storage systems prevalent
among the programming community are often too simple to match the
reality of such systems. For instance, developers commonly assume
relational databases, such as MySQL and PostgreSQL, to be
ACID-compliant with Serializable transactions, whereas in reality they
are not~\cite{bailishotos}. To accentuate performance, database
systems routinely drop Serializability in favor of weaker
\emph{isolation} guarantees that do \emph{not} mask the concurrent
nature of database transactions. This mismatch between expectation and
reality results in subtle bugs that manifest as serious violations of
safety~\cite{poloniexbug,acidrain}. The problem is aggravated in case
of distributed ``NoSQL'' data stores, which, by construction, admit
\emph{weakly-consistent} executions that defy the commonsense of
programming.  If programmers were to effectively navigate the complex
reality of modern data storage systems, novel reasoning techniques and
automated tool support are essential. Such tools and techniques should
necessarily be based on sound proof theory and accurate formal models
of modern data storage systems, which are among the major contribution
of my research.  In~\cite{popl18}, I present the first ever high-level
formal semantics of \emph{weak isolation} as it occurs in the real
world. Based on this semantics, I develop a deductive proof system
that admits rigorous proofs of correctness for database-backed
applications composed of weakly-isolated transactions.  In a follow-up
to this work~\cite{oopsla18}, I extend the formal treatment to
distributed ``NoSQL'' data stores and their applications composed of
weakly-consistent operations. A distinguishing feature of both the
formalisms is their remarkable simplicity notwithstanding their
comprehensive treatment of complex real-world data systems. Such
simplicity is indeed what makes the formal reasoning to be automated
and applied at scale to identify concurrency-related vulnerabilities
and verify correctness of realistic (distributed) database
applications, as shall be explained in the next subsection.

My work in theoretical foundations extends to data processing systems
as well. In~\cite{ecoop18}, I demonstrate how stream processing
systems (Microsoft's Naiad in particular) can take advantage of
GC-free memory regions to significantly increase overall
throughput of the system without compromising memory safety.
The region type system I present in the paper, albeit grounded in
sound theory, is an exercise in pragmatism for it strikes a fine
balance between static type checking and lightweight run-time checks
to ensure that the type system does not ``get in the way'' of Naiad's
expert developers. Such productive collaborations with systems and
database researchers is, I believe, how my research can have the most
impact going forward.

\section*{Automatic Verification and Synthesis for Data-Centric
Applications}

My theoretical work in~\cite{popl18, oopsla18, ecoop18} gave me an
excellent framework within which I could develop novel techniques in
automated verification, modelchecking, type inference, and program
synthesis for data-centric applications. Building on the proof system
in~\cite{popl18}, I built a tool called \acidifier that can
automatically analyze MySQL and PostgreSQL applications with
weakly-isolated transactions to either detect concurrency-related
vulnerabilities or certify them correct.  \acidifier can  also infer
the weakest (hence most scalable) isolation configurations at which it
is safe to run these applications. For instance, for the TPC-C OLTP
benchmark, \acidifier discovered that none of the transactions need to
be run at Serializable isolation in order to guarantee the correct
behavior, which turned out to be accurate. Another tool I built,
called \qnine is based on a novel operational semantics of
weakly-consistent replicated data stores I present
in~\cite{oopsla18}, and extends automated (bounded) verification to
distributed ``NoSQL'' applications. \qnine analyzes distributed
applications under a finite concurrency bound by reducing the
application logic and data store semantics into set-theoretic
constraints that can be very efficiently solved using off-the-shelf
SMT solvers. This approach helped \qnine discover a range of
weak-consistency bugs in large-scale applications, including the
distributed variants of TPC-C and TPC-E OLTP benchmarks in just under
two minutes, and subsequently repair the applications through
selective consistency strengthening.

Beyond verification, my work also spans synthesis of replicated data
types (RDTs), such as replicated queues and document trees, which are
the building blocks of an emerging class of peer-to-peer and
interactive distributed applications, including blockchains and
collaborative editors. The theoretical foundations were laid
in~\cite{icfp14}, where I show that any functional data structure can
be reduced to a collection of \emph{structural relations} for
which automatic reasoning is decidable. In~\cite{icfp14}, I built a
dependent type system called \catalyst that performs the
aforementioned relational reasoning to automatically verify the
correctness of data structure transformations, such as AST
transformations in a compiler. The ideas were further developed
in~\cite{oopsla19} to automatically \emph{synthesize} data structure
transformations, in particular a \emph{merge} transformation that
merges concurrent versions of a given data type in the context of
their common ancestor. This crucial result made it possible to
automatically \emph{derive} replicated data types from first
principles, thus providing a simpler and more principled alternative
to the currently-popular variants of replicated data types that insist
on commutativity of operations (i.e., CRDTs).  The synthesis approach
was used in practice to automatically derive a standard library of \emph{mergeable}
replicated data types (MRDTs) from ordinary OCaml data types, which
were then used to build decentralized (peer-to-peer) variants of large
OLTP benchmarks~\cite{oopsla19}.

\section*{Novel Programming Models for Data-Centric Computation}

Another focal point of my research work is the development of novel
language abstractions and programming models that better suit the
nature of data-centric computing. I made two major contributions
towards this end. In~\cite{pldi15}, my collaborators and I present a
Haskell domain-specific language (DSL) called \quelea that liberates
application development on weakly-consistent ``NoSQL'' stores
from the Von Neumann style. \quelea starts with a clean-slate design
where, unlike the Von Neumann model, \emph{no} order is guaranteed
between any two operations, i.e., no consistency is assumed
whatsoever. Instead, applications are required to \emph{declare} their
consistency requirements as a series of high-level ordering
constraints on their operations, which will then be then be compiled
down to efficient low-level code that handles systems and networking
concerns (analogous to how declarative Haskell programs are compiled
to efficient low-level imperative code by GHC). As we demonstrate
in~\cite{pldi15}, \quelea drastically reduces the effort required to
build highly-scalable distributed applications on top of off-the-shelf
NoSQL data stores, e.g., Cassandra.

The second contribution is an OCaml DSL, named \carmot, for building
peer-to-peer decentralized applications~\cite{snapl19,oopsla19}. Key
to \carmot is the observation that a distributed computation (on data)
is not fundamentally different from the collaborative development (of
source code), hence familiar version control protocols, e.g, Git,
which have been successfully used to manage the latter, can be
profitably employed in the context of the former. Indeed, \carmot lets
programmers orchestrate a distributed computation very much like a Git
workflow, namely by forking branches, committing concurrent versions,
and periodically merging versions to keep the branches in sync. As we
demonstrate in~\cite{snapl19,oopsla19}, the novel version
control-inspired programming model of \carmot is indispensable in
bringing the benefits of Mergeable Replicated Data Types (MRDTs), in
particular automatic synthesis, to fully-decentralized data-centric
applications, taking us one step closer to the goal of automatically
deriving such applications from their sequential counterparts.

\section*{Future Work}

I see many opportunities for future research in data-centric
computing, which I would like to pursue in collaboration with
researchers in Systems, Databases, Cryptography, and Machine Learning.
I sketch a few possibilities below.

{\itshape\color{MidnightBlue}Formal models of complex software systems}
In~\cite{popl18} and~\cite{oopsla18}, I demonstrate the utility of
employing simple formal models of (distributed) database systems in
lieu of their complex implementations to reason about application
safety. I see this approach being useful in the context of file
systems, lock-free data structures, data processing libraries --
basically any real-world software system that accrues code and
optimizations from decades of engineering effort, and becomes too
complex to be analyzed directly by the verification tools tasked with
establishing the safety of their clients. Building high-fidelity
formal models of such systems, however, requires considerable domain
knowledge, which is often scattered through API and informal
documentation, unit tests, configuration files, logs, and commit
messages. Mining structured knowledge from these sources, and (mostly)
automating the task of synthesizing formal models that act as proxies
for these systems during verification is a challenging-yet-worthwhile
exercise, and one that I plan to take up in near future.

{\itshape\color{MidnightBlue}Programming models for
fully-decentralized and privacy-conscious computing}  Users of
internet services are becoming increasingly vocal about their
(justifiable) privacy concerns on the internet. Ambitious initiatives
have sought to address this issue by promoting full decentralization in
application and network protocol design (e.g., Blockchains and IPFS),
and by putting people in charge of their data in their own premises
(e.g., TRVE Data).  Well-intended as they may be, such initiatives
nonetheless have systemic issues for there are currently no languages
and tools that can bridge the wide chasm between the limited human
cognition of the programmer, and planet-scale workings of the
underlying system. In~\cite{snapl19}, I show that a distributed
programming model inspired by version control workflow can
significantly reduce the cognitive burden involved in building highly
decentralized applications. However, the questions about performance
(at internet-scale) and privacy are yet unanswered, and addressing
them may require a significant re-think of how computation should be
performed over the internet. I would like to collaborate with Systems,
Database, and Cryptography researchers towards answering this
question.

{\itshape\color{MidnightBlue} Distributed vector representations of
data-centric computations} Following the runaway success of
``word2vec'' in natural language processing, there have been attempts
to build distributed vector representations of code snippets
(``code2vec'') with the aim of comparing and predicting the semantic
properties of code. While the approach is promising, the success so
far has been limited owing to the high semantic complexity of
general-purpose computations in a Turing-complete language compared to
words in a natural language. I believe more success can be had by
restricting our attention to data-centric computations of which most
are expressible in a Turing-\emph{incomplete} language such as Datalog (A
case in the point is SQL, which is a subset of Datalog. Another
example is the TPC-C benchmark we used in evaluating
\acidifier~\cite{popl18}, which was completely written in a
Turing-incomplete OCaml DSL). I am therefore interested in evaluating
the possibility of building distributed vector representations of
data-centric computations expressed in (a variant of) Datalog --
``Datalog2Vec'', in collaboration with researchers in Databases, 
Machine Learning, and NLP.

